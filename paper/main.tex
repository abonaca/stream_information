% This file is part of the stream_information project.
% Copyright 2017 the authors. All rights reserved.

% # style notes
% - it is Cram\'er--Rao not Cram\'er-Rao. And yet Fisher-matrix not Fisher--matrix.

\documentclass[modern]{aastex61}

%%\hypersetup{linkcolor=red,citecolor=green,filecolor=cyan,urlcolor=magenta}
\received{not yet; THIS IS A DRAFT}
%\revised{not yet}
%\accepted{not yet}
%% Adds "Submitted to " the arguement.
%\submitjournal{ApJ}

\shorttitle{information in stellar streams}
\shortauthors{Bonaca et al.}

\begin{document}

\title{The information content in cold stellar streams}

\correspondingauthor{Ana Bonaca}
% \email{whatevs}

\author{Ana Bonaca}
\affil{Harvard--Smithsonian Center for Astrophysics}

\author[0000-0003-2866-9403]{David W. Hogg}
\affiliation{Center for Cosmology and Particle Physics,
Department of Physics,
New York University}
\affiliation{Center for Data Science,
New York University}
\affiliation{Flatiron Institute, Simons Foundation}
\affiliation{Max-Planck-Institut f\"ur Astronomie, Heidelberg}

\begin{abstract}
Cold stellar streams---produced by tidal disruptions of globular
clusters (or equivalent)---are long-lived, coherent dynamical features
in the stellar halo of the Milky Way.
They have delivered precise information about the mass distribution or
gravitational potential, including constraints on the shape of the
dark-matter halo.
Because of their different positions in phase space, different ages,
and different levels of observational scrutiny, different streams tell
us different things about the Galaxy.
Here we employ a Cram\'er--Rao (CRLB) or Fisher-matrix approach to
understand the quantitative information content in the known
streams (Pal5, GD-1, Styx, [full list here]).
This approach depends on the existence of a generative model of
stellar streams, which we have developed previously, and which permits
easy calculation of derivatives of predicted stream properties with
respect to Galaxy model parameters.
We find that, in simple, static, analytic models of the Milky Way,
streams XX and YY contain the most information about the dark-matter
shape.
For any individual stream, there are near-degeneracies between
dark-matter halo properties and other parameters, including the mass
of the Large Magellanic Cloud, the total dark-matter mass, and other
potential parameters, but we find that simultaneous fitting of multiple
streams ought to precisely constrain all parameters well.
The CRLB on any one parameter depends strongly on the model freedom;
as we permit more potential freedom, the information about, say, halo
triaxiality, reduces.
We perform experiments to demonstrate this using some potential basis
functions that permit great freedom in the potential on intermediate
scales.
The CRLB formalism also permits us to assess the value of future
measurements of stellar velocities, distances, and proper motions. We
make some comments about the information value of various new
observations that could be made of particular known streams.
\end{abstract}

\keywords{foo --- bar --- hello --- world}

\section{Introduction} \label{sec:intro}

Hello World!

\end{document}

% End of file `sample61.tex'.
